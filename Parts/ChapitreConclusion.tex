\chapter{Conclusion et perspectives}

Pouvoir se déplacer de manière simple et sans connaissances d'une ville est un réel besoin pour les personnes. Notre travail a été de mettre en place un outil permettant de trouver un chemin optimal en utilisant les transports en commun.
Nous nous sommes basés pour cela, sur une des technologies les plus utilisées actuellement, à savoir les applications web.\newline
Notre application permet essentiellement à partir de deux lieux saisis par l'utilisateur (Départ et arrivée), de retourner un chemin détaillé en prenant en compte plusieurs critères définis par l'utilisateur (plus court chemin, moins de marche...).\newline
Nous avons divisé la résolution de ce problème en 4 parties:

\begin{description}
\item[Conception et stockage des données:] après avoir collecté les données, traiter et ajouter d'autres, nous les avons représenté sous forme de graphe: chaque nœud représente une station, et chaque arc entre deux stations représente le passage d'un transport.\newline
Nous avons utilisé une base de donnée orientée graphe pour stocker ce graphe pré-construit, et bénéficier de meilleures performances dans les différentes opérations de lecture/écriture.
\item[Service Web:] nous avons réalisé un service web full-REST qui permet principalement de prendre des requêtes de recherche de chemin avec différentes critères et de répondre avec une liste de suggestions. L'API peut aussi de traiter différentes requêtes d'ajout/modification de lignes et stations.
\item[Interface Web:]  nous avons implémenté une interface client, où l'utilisateur peut entrer sa station de départ et d'arrivée, ainsi que les facteurs et le moyens de transports qu'il souhaite.
Ensuite visualiser le résultat sous forme de étapes avec informations (temps, prix et distance).

\item[Interface administrateur:] nous avons implémenté également une interface administrateur qui communique avec l'API pour l'ajout et modification des lignes de transport.

\end{description}


Lors de ce projet, nous avons fait face à certaines difficultés qui ont compromis la mise en place d'une version complète de l'application. à citer en premier lieu la spécificité de nos lignes, le manque de données et la non disponibilité des données GPS qui ont bloqué tout l'aspect visualisation sur une map.\newline\newline
Ayant désormais une base solide de l'application, nous pouvons nous focaliser sur les idées et objectifs suivants :
\begin{itemize}
	\item \textbf{Compléter les données:} en introduisant la liste complète des lignes et informations de transport pour avoir une application utilisable.
	
	\item \textbf{Enrichir les propriétés des stations: }en ajoutant des informations tel que :
		\begin{itemize}
			\item \textbf{Disponibilité:} possibilité de rendre les stations en cas de travaux (ou autres perturbations) non actives pour quelques jours.
			\item \textbf{Multiples coordonnées :} pour pouvoir donner avec plus de précision la position des arrêts à prendre, du fait que dans certains cas, les arrêts pour différentes directions sont assez éloignés.
		\end{itemize}
		
	\item \textbf{Ajouter la fonctionnalité de recherche du plus proche arrêt: } pour permettre de trouver un chemin à partir de n'importe quel point : cette fonctionnalité nécessite d'implémenter des \emph{opérations spatiales} sur le graphe. Par ailleurs, Neo4j permet déjà d'intégrer ces opérations facilement grâce à des librairies tel que \textbf{Neo4j Spatial}, et donc il ne sera pas nécessaire d'utiliser d'autres SGBD pour y arriver.

	\item \textbf{Améliorer l'algorithme de recherche de chemin :}  implémenter l'algorithme A* en utilisant des heuristiques adéquate, il est aussi possible d'implémenter différentes heuristiques pour différents scénarios.\newline
	Nous pourrions aussi envisager certaines solutions de pré-traitement qui permettent de réduire les calculs de chaque recherche.
	\item \textbf{Ajouter des fonctionnalités supplémentaires à l'application administrateur}.
	\item \textbf{Ajouter des fonctionnalités supplémentaires à l'application client,} par exemple:
		\begin{itemize}
			\item Intégrer des cartes interactives permettant de visualiser les stations parcourues en temps réel, ou de télécharger les instructions en PDF pour les utiliser ultérieurement.
			\item Intégrer dans l'interface les futures fonctionnalités : trouver la station plus proche en marquant sur carte, prendre en compte la météo et l'heure (avec éventuellement un mode nuit).
			\item Améliorer la version mobile de l'application web ou une une application Android pour plus d'interactivité.
			\item Compléter la page d'information en affichant les lignes sur carte, les informations sur les horaires et les numéros de taxis.
			\item Ajouter la possibilité de créer des comptes utilisateurs pour pouvoir enregistrer des lieux personnalisés (Maison, école,..etc.).
			\item Ajouter un espace signalisation pour permettre aux utilisateurs de poster un changement dans les lignes ou une information incorrecte afin d'assurer l'intégrité et la mise à jour des informations.
		\end{itemize}
\end{itemize}